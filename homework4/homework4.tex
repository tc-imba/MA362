\documentclass[11pt,a4paper]{article}
\usepackage{../ma362}
\semester{Fall}
\year{2019}
\subtitlenumber{4}
\author{刘逸灏 (515370910207)}

\begin{document}
\maketitle

\section{二/11}
\begin{problem}
试证
\begin{enumerate}
  \item $\overline{e^z}=e^{\bar{z}}$;
        \addtocounter{enumi}{1}
  \item $\overline{\cos z}=\cos\bar{z}.$
\end{enumerate}
\end{problem}

\subsection*{(1)}
设$z=x+yi$
$$\overline{e^z}=\overline{e^{x+yi}}=\overline{e^x(\cos y+i\sin y)}=e^x\overline{\cos y+i\sin y}=e^x(\cos y-i\sin y)=e^{x-yi}=e^{\bar{z}}.$$

\subsection*{(3)}
$$\overline{\cos z}=\overline{\frac{1}{2}(e^{iz}+e^{-iz})}=\frac{1}{2}\overline{e^{iz}}+\frac{1}{2}\overline{e^{-iz}}=\frac{1}{2}e^{i\bar{z}}+\frac{1}{2}e^{-i\bar{z}}=\cos\bar{z}.$$

\section{二/13}
\begin{problem}
试求下面各式之值
\begin{enumerate}
  \item $e^{3+i}$;
  \item $\cos(1-i).$
\end{enumerate}
\end{problem}

\subsection*{(1)}
$$e^{3+i}=e^3(cos 1+i\sin 1)=e^3\cos 1+ie^3\sin 1.$$

\subsection*{(2)}
$$\cos(1-i)=\frac{1}{2}e^{i(1-i)}+\frac{1}{2}e^{-i(1-i)}=\frac{1}{2}[e(\cos 1+i\sin 1)+e^{-1}(\cos 1-i\sin 1)]=(e+e^{-1})\cos 1+i(e-e^{-1})\sin 1.$$

\section{二/20}
\begin{problem}
试解方程:
\begin{enumerate}
  \item $e^z=1+\sqrt{3}i$;
  \item $\ln z=\dfrac{\pi i}{2}$;
  \item $1+e^z=0$.
\end{enumerate}
\end{problem}

\subsection*{(1)}
$$z=\ln(1+\sqrt{3})=\ln|1+\sqrt{3}|+i\arg(1+\sqrt{3})+2k\pi i=\ln 2+\left(2k+\frac{1}{3}\right)\pi i,\quad k\in Z.$$

\subsection*{(2)}
$$z=e^{\frac{\pi i}{2}}=\cos\frac{\pi}{2}+i\sin\frac{\pi}{2}=i.$$

\subsection*{(3)}
$$e^z=-1,$$
$$z=\ln(-1)=\ln|1|+i\arg(-1)+2k\pi i=(2k+1)\pi i,\quad k\in Z.$$

\section{二/21}
\begin{problem}
设$z=re^{i\theta}$, 试证
$$\Re[\ln(z-1)]=\frac{1}{2}\ln(1+r^2-2r\cos\theta).$$
\end{problem}
\begin{align*}
  \Re[\ln(z-1)] & =\ln|z-1|=\ln|re^{i\theta}-1|                             \\
                & =\ln|r\cos\theta+ir\sin\theta-1|                          \\
                & =\ln\sqrt{r^2\cos^2\theta-2r\cos\theta+1+r^2\sin^2\theta} \\
                & =\frac{1}{2}\ln(1+r^2-2r\cos\theta).
\end{align*}

\section{二/22}
\begin{problem}
设$w=\sqrt[3]{z}$确定在从原点$z=0$起沿正实轴割破了的$z$平面上, 并且$w(i)=-i$, 试求$w(-i)$之值.
\end{problem}
设
$$w(z)=\sqrt[3]{z}=\sqrt[3]{|z|}e^{i\frac{\arg z+2k\pi}{3}},\quad k=0,1,2.$$
代入得
$$w(i)=\sqrt[3]{|i|}e^{i\frac{\arg i+2k\pi}{3}}=e^{i\frac{(2k+1/2)\pi}{3}}=e^{i\frac{3}{2}\pi}=-i,$$
$$\frac{2k+1/2}{3}\pi=\frac{3}{2}\pi\Longrightarrow k=2.$$
故
$$w(z)=\sqrt[3]{|z|}e^{i\frac{\arg z+4\pi}{3}},$$
$$w(-i)=\sqrt[3]{|-i|}e^{i\frac{\arg(-i)+4\pi}{3}}=e^{i\frac{(4+3/2)\pi}{3}}=e^{i\frac{11}{6}\pi}=\cos\frac{11}{6}\pi+i\sin\frac{11}{6}\pi=\frac{\sqrt{3}}{2}-\frac{1}{2}i.$$

\section{二/23}
\begin{problem}
设$w=\sqrt[3]{z}$确定在从原点$z=0$起沿负实轴割破了的$z$平面上, 并且$w(-2)=-\sqrt[3]{2}$(这是边界上岸点对应的函数值), 试求$w(i)$之值.
\end{problem}
设
$$w(z)=\sqrt[3]{z}=\sqrt[3]{|z|}e^{i\frac{\arg z+2k\pi}{3}},\quad k=0,1,2.$$
代入得
$$w(-2)=\sqrt[3]{|-2|}e^{i\frac{\arg(-2)+2k\pi}{3}}=\sqrt[3]{2}e^{i\frac{(2k+1)\pi}{3}}=\sqrt[3]{2}e^{i\pi}=-\sqrt[3]{2},$$
$$\frac{2k+1}{3}\pi=\pi\Longrightarrow k=1.$$
故
$$w(z)=\sqrt[3]{|z|}e^{i\frac{\arg z+2\pi}{3}},$$
$$w(i)=\sqrt[3]{|i|}e^{i\frac{\arg i+2\pi}{3}}=e^{i\frac{(2+1/2)\pi}{3}}=e^{i\frac{5}{6}\pi}=\cos\frac{5}{6}\pi+i\sin\frac{5}{6}\pi=-\frac{\sqrt{3}}{2}+\frac{1}{2}i.$$


\end{document}
