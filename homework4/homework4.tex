\documentclass[11pt,a4paper]{article}
\usepackage{../ma362}
\semester{Fall}
\year{2019}
\subtitlenumber{4}
\author{刘逸灏 (515370910207)}

\begin{document}
\maketitle

\section{二(一)/11}
\begin{problem}
试证
\begin{enumerate}
  \item $\overline{e^z}=e^{\bar{z}}$;
        \addtocounter{enumi}{1}
  \item $\overline{\cos z}=\cos\bar{z}.$
\end{enumerate}
\end{problem}

\subsection*{(1)}
设$z=x+yi$
$$\overline{e^z}=\overline{e^{x+yi}}=\overline{e^x(\cos y+i\sin y)}=e^x\overline{\cos y+i\sin y}=e^x(\cos y-i\sin y)=e^{x-yi}=e^{\bar{z}}.$$

\subsection*{(3)}
$$\overline{\cos z}=\overline{\frac{1}{2}(e^{iz}+e^{-iz})}=\frac{1}{2}\overline{e^{iz}}+\frac{1}{2}\overline{e^{-iz}}=\frac{1}{2}e^{i\bar{z}}+\frac{1}{2}e^{-i\bar{z}}=\cos\bar{z}.$$

\section{二(一)/13}
\begin{problem}
试求下面各式之值
\begin{enumerate}
  \item $e^{3+i}$;
  \item $\cos(1-i).$
\end{enumerate}
\end{problem}

\subsection*{(1)}
$$e^{3+i}=e^3(cos 1+i\sin 1)=e^3\cos 1+ie^3\sin 1.$$

\subsection*{(2)}
$$\cos(1-i)=\frac{1}{2}e^{i(1-i)}+\frac{1}{2}e^{-i(1-i)}=\frac{1}{2}[e(\cos 1+i\sin 1)+e^{-1}(\cos 1-i\sin 1)]=(e+e^{-1})\cos 1+i(e-e^{-1})\sin 1.$$

\section{二(一)/20}
\begin{problem}
试解方程:
\begin{enumerate}
  \item $e^z=1+\sqrt{3}i$;
  \item $\ln z=\dfrac{\pi i}{2}$;
  \item $1+e^z=0$.
\end{enumerate}
\end{problem}

\subsection*{(1)}
$$z=\ln(1+\sqrt{3})=\ln|1+\sqrt{3}|+i\arg(1+\sqrt{3})+2k\pi i=\ln 2+\left(2k+\frac{1}{3}\right)\pi i,\quad k\in Z.$$

\subsection*{(2)}
$$z=e^{\frac{\pi i}{2}}=\cos\frac{\pi}{2}+i\sin\frac{\pi}{2}=i.$$

\subsection*{(3)}
$$e^z=-1,$$
$$z=\ln(-1)=\ln|1|+i\arg(-1)+2k\pi i=(2k+1)\pi i,\quad k\in Z.$$

\section{二(一)/21}
\begin{problem}
设$z=re^{i\theta}$, 试证
$$\Re[\ln(z-1)]=\frac{1}{2}\ln(1+r^2-2r\cos\theta).$$
\end{problem}
\begin{align*}
  \Re[\ln(z-1)] & =\ln|z-1|=\ln|re^{i\theta}-1|                             \\
                & =\ln|r\cos\theta+ir\sin\theta-1|                          \\
                & =\ln\sqrt{r^2\cos^2\theta-2r\cos\theta+1+r^2\sin^2\theta} \\
                & =\frac{1}{2}\ln(1+r^2-2r\cos\theta).
\end{align*}

\section{二(一)/22}
\begin{problem}
设$w=\sqrt[3]{z}$确定在从原点$z=0$起沿正实轴割破了的$z$平面上, 并且$w(i)=-i$, 试求$w(-i)$之值.
\end{problem}
设
$$w(z)=\sqrt[3]{z}=\sqrt[3]{|z|}e^{i\frac{\arg z+2k\pi}{3}},\quad k=0,1,2.$$
代入得
$$w(i)=\sqrt[3]{|i|}e^{i\frac{\arg i+2k\pi}{3}}=e^{i\frac{(2k+1/2)\pi}{3}}=e^{i\frac{3}{2}\pi}=-i,$$
$$\frac{2k+1/2}{3}\pi=\frac{3}{2}\pi\Longrightarrow k=2.$$
故
$$w(z)=\sqrt[3]{|z|}e^{i\frac{\arg z+4\pi}{3}},$$
$$w(-i)=\sqrt[3]{|-i|}e^{i\frac{\arg(-i)+4\pi}{3}}=e^{i\frac{(4+3/2)\pi}{3}}=e^{i\frac{11}{6}\pi}=\cos\frac{11}{6}\pi+i\sin\frac{11}{6}\pi=\frac{\sqrt{3}}{2}-\frac{1}{2}i.$$

\section{二(一)/23}
\begin{problem}
设$w=\sqrt[3]{z}$确定在从原点$z=0$起沿负实轴割破了的$z$平面上, 并且$w(-2)=-\sqrt[3]{2}$(这是边界上岸点对应的函数值), 试求$w(i)$之值.
\end{problem}
设
$$w(z)=\sqrt[3]{z}=\sqrt[3]{|z|}e^{i\frac{\arg z+2k\pi}{3}},\quad k=0,1,2.$$
代入得
$$w(-2)=\sqrt[3]{|-2|}e^{i\frac{\arg(-2)+2k\pi}{3}}=\sqrt[3]{2}e^{i\frac{(2k+1)\pi}{3}}=\sqrt[3]{2}e^{i\pi}=-\sqrt[3]{2},$$
$$\frac{2k+1}{3}\pi=\pi\Longrightarrow k=1.$$
故
$$w(z)=\sqrt[3]{|z|}e^{i\frac{\arg z+2\pi}{3}},$$
$$w(i)=\sqrt[3]{|i|}e^{i\frac{\arg i+2\pi}{3}}=e^{i\frac{(2+1/2)\pi}{3}}=e^{i\frac{5}{6}\pi}=\cos\frac{5}{6}\pi+i\sin\frac{5}{6}\pi=-\frac{\sqrt{3}}{2}+\frac{1}{2}i.$$

\section{二(一)/24}
\begin{problem}
试求$(1+i)^i$和$3^i$之值.
\end{problem}

$$(1+i)^i=e^{i\text{Ln}(1+i)}=e^{i[\ln|1+i|+i\arg(1+i)+2k\pi i]}=e^{-(2k+1/4)\pi}e^{i\ln\sqrt{2}},\quad k\in Z.$$
$$3^i=e^{i\text{Ln}3}=e^{i(\ln|3|+i\arg3+2k\pi i)}=e^{-2k\pi}e^{i\ln 3},\quad k\in Z.$$

\section{三(一)/8}
\begin{problem}
由积分$\displaystyle\int_C\frac{dz}{z+2}$之值证明
$$\int_0^\pi\frac{1+2\cos\theta}{5+4\cos\theta}d\theta=0,$$
其中$C$取单位圆周$|z|=1$.
\end{problem}

设$f(z)=\dfrac{1}{z+2}$, $D$为$C$围成的域, 则$f\in H(D)\cap C(\overline{D})$, 根据柯西定理得
$$\int_C\frac{dz}{z+2}=0.$$
设$z=e^{i\theta}$
\begin{align*}
  \int_C\frac{dz}{z+2}
   & =\int_C\frac{de^{i\theta}}{e^{i\theta}+2}                                                                                            \\
   & =\int_0^{2\pi}\frac{ie^{i\theta}}{e^{i\theta}+2}d\theta                                                                              \\
   & =\int_0^{2\pi}\frac{i\cos\theta-\sin\theta}{\cos\theta+2+i\sin\theta}d\theta                                                         \\
   & =\int_0^{2\pi}\frac{(i\cos\theta-\sin\theta)(\cos\theta+2-i\sin\theta)}{(\cos\theta+2+i\sin\theta)(\cos\theta+2-i\sin\theta)}d\theta \\
   & =\int_0^{2\pi}\frac{i(1+2\cos\theta)-2\sin\theta}{5+4\cos\theta}d\theta=0.
\end{align*}
$$\int_0^{2\pi}\frac{1+2\cos\theta}{5+4\cos\theta}d\theta=\int_0^{2\pi}\frac{-2\sin\theta}{5+4\cos\theta}d\theta=0.$$
由于$\dfrac{1+2\cos\theta}{5+4\cos\theta}$是周期为$2\pi$的周期函数
$$\int_0^{2\pi}\frac{1+2\cos\theta}{5+4\cos\theta}d\theta=\int_{-\pi}^{\pi}\frac{1+2\cos\theta}{5+4\cos\theta}d\theta=0.$$
由于$\dfrac{1+2\cos\theta}{5+4\cos\theta}$是偶函数
$$\int_0^{\pi}\frac{1+2\cos\theta}{5+4\cos\theta}d\theta=\frac{1}{2}\int_{-\pi}^{\pi}\frac{1+2\cos\theta}{5+4\cos\theta}d\theta=0.$$

\section{三(一)/9}
\begin{problem}
计算$(C:|z|=2)$:
\begin{enumerate}
  \item $\displaystyle\int_C\frac{2z^2-z+1}{z-1}dz$;
  \item $\displaystyle\int_C\frac{2z^2-z+1}{(z-1)^2}dz$;
\end{enumerate}
\end{problem}

\subsection*{(1)}
设$f(z)=2z^2-z+1$, $D$为$C$围成的域, 则$f\in H(D)\cap C(\overline{D})$, 根据柯西积分定理得
$$f(z)=\frac{1}{2\pi i}\int_C\frac{f(\xi)}{\xi-z}d\xi,$$
$$\int_C\frac{2z^2-z+1}{z-1}dz=\int_C\frac{f(\xi)}{\xi-1}d\xi=2\pi i\cdot f(1)=4\pi i.$$

\subsection*{(1)}
设$f(z)=2z^2-z+1$, $D$为$C$围成的域, 则$f\in H(D)\cap C(\overline{D})$, 根据柯西积分定理得
$$f'(z)=4z-1=\frac{1!}{2\pi i}\int_C\frac{f(\xi)}{(\xi-z)^2}d\xi,$$
$$\int_C\frac{2z^2-z+1}{(z-1)^2}dz=\int_C\frac{f(\xi)}{(\xi-1)^2}d\xi=2\pi i\cdot f'(1)=6\pi i.$$

\section{三(一)/10}
\begin{problem}
计算积分:
$$\int_{C_j}\frac{\sin\dfrac{\pi}{4}z}{z^2-1}dz\quad (j=1,2,3),$$
\begin{enumerate}
  \item $C_1:|z+1|=\dfrac{1}{2}$;
  \item $C_2:|z-1|=\dfrac{1}{2}$;
  \item $C_3:|z|=2$.
\end{enumerate}
\end{problem}

设$f(z)=\sin\dfrac{\pi}{4}z$, $D_j$为$C_j$围成的域, 则$f\in H(D_j)\cap C(\overline{D_j})$, 根据柯西积分定理得
$$\int_{C_j}\frac{\sin\dfrac{\pi}{4}z}{z^2-1}dz=\int_{C_j}\sin\dfrac{\pi}{4}z\cdot\frac{1}{2}\left(\frac{1}{z-1}-\frac{1}{z+1}\right)dz=\frac{1}{2}\int_{C_j}\frac{\sin\dfrac{\pi}{4}z}{z-1}dz-\frac{1}{2}\int_{C_j}\frac{\sin\dfrac{\pi}{4}z}{z+1}dz.$$

\subsection*{(1)}
$1$在$C_1$内, $-1$在$C_1$外, 根据柯西定理得
$$\int_{C_1}\frac{\sin\dfrac{\pi}{4}z}{z+1}dz=0.$$
根据柯西积分定理得
$$f(z)=\frac{1}{2\pi i}\int_{C_1}\frac{f(\xi)}{\xi-z}d\xi,$$
$$\int_{C_1}\frac{\sin\dfrac{\pi}{4}\xi}{\xi-1}d\xi=\int_{C_1}\frac{f(\xi)}{\xi-1}d\xi=2\pi i\cdot f(1)=\sqrt{2}\pi i,$$
$$\int_{C_1}\frac{\sin\dfrac{\pi}{4}z}{z^2-1}dz=\frac{1}{2}\int_{C_1}\frac{\sin\dfrac{\pi}{4}z}{z-1}dz-\frac{1}{2}\int_{C_1}\frac{\sin\dfrac{\pi}{4}z}{z+1}dz=\frac{\sqrt{2}}{2}\pi i.$$

\subsection*{(2)}
$-1$在$C_2$内, $1$在$C_2$外, 根据柯西定理得
$$\int_{C_2}\frac{\sin\dfrac{\pi}{4}z}{z-1}dz=0.$$
根据柯西积分定理得
$$f(z)=\frac{1}{2\pi i}\int_{C_2}\frac{f(\xi)}{\xi-z}d\xi,$$
$$\int_{C_2}\frac{\sin\dfrac{\pi}{4}\xi}{\xi+1}d\xi=\int_{C_2}\frac{f(\xi)}{\xi+1}d\xi=2\pi i\cdot f(-1)=-\sqrt{2}\pi i,$$
$$\int_{C_2}\frac{\sin\dfrac{\pi}{4}z}{z^2-1}dz=\frac{1}{2}\int_{C_2}\frac{\sin\dfrac{\pi}{4}z}{z-1}dz-\frac{1}{2}\int_{C_2}\frac{\sin\dfrac{\pi}{4}z}{z+1}dz=\frac{\sqrt{2}}{2}\pi i.$$

\subsection*{(3)}
$-1,1$在$C_3$内, 同(1), (2)可得
$$\int_{C_3}\frac{\sin\dfrac{\pi}{4}\xi}{\xi-1}d\xi=\sqrt{2}\pi i,$$
$$\int_{C_3}\frac{\sin\dfrac{\pi}{4}\xi}{\xi+1}d\xi=-\sqrt{2}\pi i,$$
$$\int_{C_3}\frac{\sin\dfrac{\pi}{4}z}{z^2-1}dz=\frac{1}{2}\int_{C_3}\frac{\sin\dfrac{\pi}{4}z}{z-1}dz-\frac{1}{2}\int_{C_3}\frac{\sin\dfrac{\pi}{4}z}{z+1}dz=\sqrt{2}\pi i.$$

\section{三(一)/11}
\begin{problem}
求积分$$\int_C\frac{e^z}{z}dz(C:|z|=1),$$
从而证明$$\int_0^\pi e^{\cos\theta}\cos(\sin\theta)d\theta=\pi.$$
\end{problem}

设$f(z)=e^z$, $D$为$C$围成的域, 则$f\in H(D)\cap C(\overline{D})$, 根据柯西积分公式得
$$f(z)=\frac{1}{2\pi i}\int_{|\xi|=1}\frac{f(\xi)}{\xi-z}d\xi,$$
$$\int_{|z|=1}\frac{e^z}{z}dz=\int_{|\xi|=1}\frac{f(\xi)}{\xi-0}d\xi=2\pi i\cdot f(0)=2\pi i.$$

设$z=e^{i\theta}$
$$2\pi i=\int_C\frac{e^z}{z}dz=\int_C\frac{e^{e^{i\theta}}}{e^{i\theta}}de^{i\theta}=\int_0^{2\pi}\frac{ie^{i\theta}e^{e^{i\theta}}}{e^{i\theta}}d\theta=\int_0^{2\pi}ie^{e^{i\theta}}d\theta,$$
$$2\pi=\int_0^{2\pi}e^{e^{i\theta}}d\theta=\int_0^{2\pi}e^{\cos\theta+i\sin\theta}d\theta=\int_0^{2\pi}e^{\cos\theta}[\cos(\sin\theta)+i\sin(\sin\theta)]d\theta,$$
$$\int_0^{2\pi}e^{\cos\theta}\cos(\sin\theta)d\theta=2\pi,\quad \int_0^{2\pi}e^{\cos\theta}\sin(\sin\theta)d\theta=0.$$
由于$e^{\cos\theta}\cos(\sin\theta)$是周期为$2\pi$的周期函数
$$\int_0^{2\pi}e^{\cos\theta}\cos(\sin\theta)d\theta=\int_{-\pi}^{\pi}e^{\cos\theta}\cos(\sin\theta)d\theta=2\pi.$$
由于$e^{\cos\theta}\cos(\sin\theta)$是偶函数
$$\int_{0}^{\pi}e^{\cos\theta}\cos(\sin\theta)d\theta=\frac{1}{2}\int_{-\pi}^{\pi}e^{\cos\theta}\cos(\sin\theta)d\theta=\pi.$$

\section{三(一)/12}
\begin{problem}
设$C$表圆周$x^2+y^2=3$, $\displaystyle f(z)=\int_C\frac{3\xi^2+7\xi+1}{\xi-z}d\xi,$
求$f'(1+i)$.
\end{problem}

设$g(z)=3z^2+7z+1$, $D$为$C$围成的域, $1+i\in D$, 则$g\in H(D)\cap C(\overline{D})$, 根据柯西积分公式得
$$g(z)=\frac{1}{2\pi i}\int_{|\xi|=3}\frac{g(\xi)}{\xi-z}d\xi,$$
$$f(z)=2\pi i\cdot g(z)=2\pi i(3z^2+7z+1),$$
$$f'(z)=2\pi i(6z+7),$$
$$f'(1+i)=2\pi i(6+6i+7)=-12\pi+26\pi i.$$

\section{三(一)/13}
\begin{problem}
设$C:z=z(t)(\alpha\leqslant t\leqslant\beta)$为区域$D$内的光滑曲线, $f(z)$于区域$D$内单叶解析且$f'(z)\neq0$, $\omega=f(z)$将$C$映射成曲线$\Gamma$, 求证$\Gamma$亦为光滑曲线.
\end{problem}
$z=z(t)(\alpha\leqslant t\leqslant\beta)$为光滑曲线的充要条件为$\forall t_1,t_2\in[\alpha,\beta]$, $t_1\neq t_2$时, $z(t_1)\neq z(t_2)$, 且$z'(t)\neq 0$并连续于$[\alpha,\beta]$.\medskip

现要证明$\Gamma:\omega=\omega(t)=f(z(t))(\alpha\leqslant t\leqslant\beta)$为光滑曲线.

首先, $\forall t_1,t_2\in[\alpha,\beta]$, $t_1\neq t_2$时, $z(t_1)\neq z(t_2)$, 由于$f(z)$于区域$D$内单叶, $f(z(t_1))\neq f(z(t_2))$.

其次, $w'(t)=[f(z(t))]'=f'(z)z'(t)\neq 0$, 且由$f(z)$于区域$D$内解析得$f'(z)$连续于$[\alpha,\beta]$, 故$w'(t)$连续于$[\alpha,\beta]$.

综上可得$\Gamma$为光滑曲线.


\end{document}
