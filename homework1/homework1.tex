\documentclass[11pt,a4paper]{article}
\usepackage{../ma362}
\semester{Fall}
\year{2019}
\subtitlenumber{1}
\author{刘逸灏 (515370910207)}

\begin{document}
\maketitle

\section{2.1/1}

\subsection*{(i)}

对于任意$z\in\mathbf{C}$有
$$\frac{f(z+h)-f(z)}{h}=\frac{|z+h|-|z|}{h}=\frac{|z+h|^2-|z|^2}{h(|z+h|+|z|)}=\frac{z\bar{h}+\bar{z}h+|h|^2}{h(|z+h|+|z|)}.$$

当$z\neq0$时
$$\lim_{h\to0}\frac{f(z+h)-f(z)}{h}=\lim_{h\to0}\frac{z\bar{h}+\bar{z}h}{2|z|h}=\lim_{h\to0}\frac{z\bar{h}/h+\bar{z}}{2|z|}.$$

如果让$h$取实数, 则上述极限为$\dfrac{z+\bar{z}}{2|z|}$; 如果让$h$取纯虚数, 则上述极限为$\dfrac{-z+\bar{z}}{2|z|}$.

当$z=0$时
$$\lim_{h\to0}\frac{f(z+h)-f(z)}{h}=\lim_{h\to0}\frac{|h|^2}{h|h|}=\lim_{h\to0}\frac{|h|}{h}.$$

如果让$h$取实数, 则上述极限为$1$; 如果让$h$取纯虚数, 则上述极限为$-1$.

因此, 当$h\to0$时上述极限不存在, 因而在$\mathbf{C}$中处处不可微.

\subsection*{(ii)}

对于任意$z\in\mathbf{C}$有
$$\frac{f(z+h)-f(z)}{h}=\frac{|z+h|^2-|z|^2}{h}=\frac{z\bar{h}+\bar{z}h+|h|^2}{h}.$$

当$z\neq0$时
$$\lim_{h\to0}\frac{f(z+h)-f(z)}{h}=\lim_{h\to0}\frac{z\bar{h}+\bar{z}h}{h}=\lim_{h\to0}(z\bar{h}/h+\bar{z}).$$

如果让$h$取实数, 则上述极限为$z+\bar{z}$; 如果让$h$取纯虚数, 则上述极限为$-z+\bar{z}$.

当$z=0$时
$$\lim_{h\to0}\frac{f(z+h)-f(z)}{h}=\lim_{h\to0}\frac{|h|^2}{h}=0.$$

因此, 当$z\neq0$,$h\to0$时上述极限不存在, 因而仅在$z=0$处可微.

\subsection*{(iii)}

对于任意$z\in\mathbf{C}$有
$$\lim_{h\to0}\frac{f(z+h)-f(z)}{h}=\lim_{h\to0}\frac{\Re(z+h)-\Re(z)}{h}=\lim_{h\to0}\frac{\Re(h)}{h}.$$

如果让$h$取实数, 则上述极限为$1$; 如果让$h$取纯虚数, 则上述极限为$0$.

因此, 当$h\to0$时上述极限不存在, 因而在$\mathbf{C}$中处处不可微.

\subsection*{(v)}

设$f(z)=z_0$, 对于任意$z\in\mathbf{C}$有
$$\lim_{h\to0}\frac{f(z+h)-f(z)}{h}=\lim_{h\to0}\frac{z_0-z_0}{h}=0.$$

因此, 当$h\to0$时上述极限存在, 因而在$\mathbf{C}$中处处可微.

\section{2.1/2}

由$f$和$g$都在$z_0$处可微有
$$f'(z_0)=\lim_{z\to z_0}\frac{f(z)-f(z_0)}{z-z_0},$$
$$g'(z_0)=\lim_{z\to z_0}\frac{g(z)-g(z_0)}{z-z_0}.$$

由$f(z_0)=g(z_0)=0$, $g'(z_0)\neq 0$可得
$$\frac{f'(z_0)}{g'(z_0)}=\lim_{z\to z_0}\frac{f(z)-f(z_0)}{g(z)-g(z_0)}=\lim_{z\to z_0}\frac{f(z)}{g(z)}.$$

\section{2.1/4}

对于任意$z\in\mathbf{G}$有$\bar{z}\in\mathbf{D}$, 设$z+h\in\mathbf{G}$, 则$\overline{z+h}\in\mathbf{D}$.由$f$是$\mathbf{D}$上的全纯函数可得
$$f'(\bar{z})=\lim_{\bar{h}\to0}\frac{f({\overline{z+h}})-f(\bar{z})}{\bar{h}},$$
$$\overline{f'(\bar{z})}=\overline{\lim_{\bar{h}\to0}\frac{f({\overline{z+h}})-f(\bar{z})}{\bar{h}}}=\lim_{h\to0}\frac{\overline{f(\overline{z+h})-f(\bar{z})}}{h}=\lim_{h\to0}\frac{\overline{f({\overline{z+h}})}-\overline{f(\bar{z})}}{h}.$$

因此$\overline{f(\bar{z})}$是$\mathbf{G}$上的全纯函数.

\section{2.2/1}

设$f(z)=u(x,y)+iv(x,y)$, 由$f\in H(D)$可知
$$f'(z)=\frac{\partial u}{\partial x}+i\frac{\partial v}{\partial x}=\frac{\partial v}{\partial y}+i\frac{\partial u}{\partial y}=0.$$

故
$$\frac{\partial u}{\partial x}=\frac{\partial v}{\partial x}=\frac{\partial v}{\partial y}=\frac{\partial u}{\partial y}=0.$$

因此$u(x,y)$和$v(x,y)$都为常数, 因而$f$是一常数.

\section{2.2/2}

设$f(z)=u(x,y)+iv(x,y)$, 由$f\in H(D)$可知
$$f'(z)=\frac{\partial u}{\partial x}+i\frac{\partial v}{\partial x}=\frac{\partial v}{\partial y}+i\frac{\partial u}{\partial y},\quad\frac{\partial u}{\partial x}=\frac{\partial v}{\partial y},\quad\frac{\partial u}{\partial y}=-\frac{\partial v}{\partial x}.$$

\subsection*{(i)}

由$\Re f(z)$是常数可得
$$\frac{\partial u}{\partial x}=\frac{\partial u}{\partial y}=0.$$

故
$$\frac{\partial u}{\partial x}=\frac{\partial v}{\partial x}=\frac{\partial v}{\partial y}=\frac{\partial u}{\partial y}=0.$$

因此$u(x,y)$和$v(x,y)$都为常数, 因而$f$是一常数.

\subsection*{(ii)}

由$\Im f(z)$是常数可得
$$\frac{\partial v}{\partial x}=\frac{\partial v}{\partial y}=0.$$

故
$$\frac{\partial u}{\partial x}=\frac{\partial v}{\partial x}=\frac{\partial v}{\partial y}=\frac{\partial u}{\partial y}=0.$$

因此$u(x,y)$和$v(x,y)$都为常数, 因而$f$是一常数.

\subsection*{(iii)}

由$|f(z)|$是常数$k\in\mathbf{R}$可得
$$u^2(x,y)+v^2(x,y)=k^2.$$

若$k=0$易知$u(x,y)=v(x,y)=0$,$f(z)=0$.

若$k=0$且存在, 分别对$x$, $y$求偏导可得
$$2u(x,y)\frac{\partial u}{\partial x}+2v(x,y)\frac{\partial v}{\partial x}=u(x,y)\frac{\partial u}{\partial x}-v(x,y)\frac{\partial u}{\partial y}=0,$$
$$2u(x,y)\frac{\partial u}{\partial y}+2v(x,y)\frac{\partial v}{\partial y}=u(x,y)\frac{\partial u}{\partial y}+v(x,y)\frac{\partial u}{\partial x}=0.$$

解方程组可得
$$\left[u^2(x,y)+v^2(x,y)\right]\frac{\partial u}{\partial x}=\left[u^2(x,y)+v^2(x,y)\right]\frac{\partial u}{\partial y}=0,$$
$$\frac{\partial u}{\partial x}=\frac{\partial u}{\partial y}=0.$$

易知$u(x,y)$是常数, 由(i)可得$f$为常数. 因而$f$是一常数.

\subsection*{(iv)}

由$\arg f(z)$是常数$k\in\mathbf{R}$可得
$$\frac{v(x,y)}{u(x,y)}=\tan k.$$

若$\tan k=0$或无穷易知$u(x,y)$或$v(x,y)$是常数, 由(i)(ii)可得$f$为常数.

若$\tan k\neq0$且存在,分别对$x$, $y$求偏导可得
$$\frac{\partial v}{\partial x}=\tan k\frac{\partial u}{\partial x}=-\frac{\partial u}{\partial y},\quad\frac{\partial v}{\partial y}=\tan k\frac{\partial u}{\partial y}=\frac{\partial u}{\partial x}.$$

解方程组可得
$$(1+\tan^2k)\frac{\partial u}{\partial x}=(1+\tan^2k)\frac{\partial u}{\partial y}=0,$$
$$\frac{\partial u}{\partial x}=\frac{\partial u}{\partial y}=0.$$

易知$u(x,y)$是常数, 由(i)可得$f$为常数. 因而$f$是一常数.

\section{2.2/3}

设$f(z)=u(x,y)+iv(x,y)=\sqrt{xy}$, $u(x,y)=\sqrt{xy}$, $v(x,y)=0$.
$$\frac{\partial u}{\partial x}=\lim_{h\to0}\frac{u(h,0)-u(0,0)}{h}=0,$$
$$\frac{\partial u}{\partial y}=\lim_{h\to0}\frac{u(0,h)-u(0,0)}{h}=0,$$
$$\frac{\partial v}{\partial x}=\frac{\partial v}{\partial y}=0,$$
$$\frac{\partial f}{\partial\bar{z}}=\frac{1}{2}\left(\frac{\partial u}{\partial x}-\frac{\partial v}{\partial y}\right)+\frac{i}{2}\left(\frac{\partial u}{\partial y}+\frac{\partial v}{\partial x}\right)=0.$$

故$f(z)$在$z=0$处满足Cauchy-Riemann方程.

设$h=x+xi$
$$\lim_{h\to0}\frac{f(h)-f(0)}{h}=\lim_{x\to0}\frac{|x|}{x+xi}.$$

如果让$x$取正数, 则上述极限为$\dfrac{1}{1+i}$; 如果让$h$取负数, 则上述极限为$-\dfrac{1}{1+i}$.

因此, 当$h\to0$时上述极限不存在, 因而$f$在$z=0$中处不可微.

\section{2.2/4}

由$z=x+yi=r(\cos\theta+i\sin\theta)$得
$$x(r,\theta)=r\cos\theta,\quad y(r,\theta)=r\sin\theta,$$
$$\frac{\partial x}{\partial r}=\cos\theta,\quad \frac{\partial x}{\partial \theta}=-r\sin\theta,$$
$$\frac{\partial y}{\partial r}=\sin\theta,\quad \frac{\partial y}{\partial \theta}=r\cos\theta.$$

$f(z)=u(x,y)+iv(x,y)=u(r,\theta)+iv(r,\theta)$, 由链式求导法则可得
$$\frac{\partial u}{\partial r}=\frac{\partial u}{\partial x}\frac{\partial x}{\partial r}+\frac{\partial u}{\partial y}\frac{\partial y}{\partial r}=\frac{\partial u}{\partial x}\cos\theta+\frac{\partial u}{\partial y}\sin\theta,$$
$$\frac{1}{r}\frac{\partial v}{\partial\theta}=\frac{1}{r}\left(\frac{\partial v}{\partial x}\frac{\partial x}{\partial\theta}+\frac{\partial v}{\partial y}\frac{\partial y}{\partial\theta}\right)=-\frac{\partial v}{\partial x}\sin\theta+\frac{\partial v}{\partial y}\cos\theta,$$
$$\frac{\partial v}{\partial r}=\frac{\partial v}{\partial x}\frac{\partial x}{\partial r}+\frac{\partial v}{\partial y}\frac{\partial y}{\partial r}=\frac{\partial v}{\partial x}\cos\theta+\frac{\partial v}{\partial y}\sin\theta,$$
$$-\frac{1}{r}\frac{\partial u}{\partial\theta}=\frac{1}{r}\left(\frac{\partial u}{\partial x}\frac{\partial x}{\partial\theta}+\frac{\partial u}{\partial y}\frac{\partial y}{\partial\theta}\right)=\frac{\partial u}{\partial x}\sin\theta-\frac{\partial u}{\partial y}\cos\theta.$$

代入Cauchy-Riemann方程
$$\frac{\partial u}{\partial x}=\frac{\partial v}{\partial y},\quad\frac{\partial u}{\partial y}=-\frac{\partial v}{\partial x}.$$

可知
$$\frac{\partial u}{\partial r}=\frac{1}{r}\frac{\partial v}{\partial\theta},\quad \frac{\partial v}{\partial r}=-\frac{1}{r}\frac{\partial u}{\partial\theta}.$$

\section{2.2/5}

同上题, 由链式求导法则可得
$$\frac{\partial f}{\partial r}=\frac{\partial f}{\partial x}\frac{\partial x}{\partial r}+\frac{\partial f}{\partial y}\frac{\partial y}{\partial r}=\frac{\partial f}{\partial x}\cos\theta+\frac{\partial f}{\partial y}\sin\theta,$$
$$\frac{i}{r}\frac{\partial f}{\partial\theta}=\frac{i}{r}=\left(\frac{\partial f}{\partial x}\frac{\partial x}{\partial\theta}+\frac{\partial f}{\partial y}\frac{\partial y}{\partial\theta}\right)=-\frac{\partial f}{\partial x}i\sin\theta+\frac{\partial f}{\partial y}i\cos\theta.$$

代入可知
$$\frac{1}{2}e^{i\theta}\left(\frac{\partial f}{\partial r}+\frac{i}{r}\frac{\partial f}{\partial\theta}\right)=
  \frac{1}{2}e^{i\theta}\left[\frac{\partial f}{\partial x}(\cos\theta-i\sin\theta)+\frac{\partial f}{\partial y}(\sin\theta+i\cos\theta)\right]=
  \frac{1}{2}\left(\frac{\partial f}{\partial x}+i\frac{\partial f}{\partial y}\right)=\frac{\partial f}{\partial\bar{z}},$$
$$\frac{1}{2}e^{-i\theta}\left(\frac{\partial f}{\partial r}-\frac{i}{r}\frac{\partial f}{\partial\theta}\right)=
  \frac{1}{2}e^{-i\theta}\left[\frac{\partial f}{\partial x}(\cos\theta+i\sin\theta)+\frac{\partial f}{\partial y}(\sin\theta-i\cos\theta)\right]=
  \frac{1}{2}\left(\frac{\partial f}{\partial x}-i\frac{\partial f}{\partial y}\right)=\frac{\partial f}{\partial z}.$$



\section{2.2/8}

由$f\in H(D)$可知
$$\frac{\partial f}{\partial z}=f'(z),\quad \frac{\partial f}{\partial\bar{z}}=0.$$

由$\overline{f(\bar{z})}\in H(D)$由此可知
$$\overline{f(\overline{z_0+\Delta z})}-\overline{f(\overline{z_0})}=\frac{\partial\bar{f}}{\partial z}\Delta z+ \frac{\partial\bar{f}}{\partial\bar{z}}\overline{\Delta z}+o(|\Delta z|).$$
$$\frac{\partial\bar{f}}{\partial\bar{z}}=\overline{f'(z)},\quad\frac{\partial\bar{f}}{\partial z}=0.$$

故
\begin{align*}
   & \quad \left(\frac{\partial^2}{\partial x^2}+\frac{\partial^2}{\partial y^2}\right)|f(z)|^p          \\
   & = \Delta|f(z)\overline{f(z)}|^{p/2}                                                                 \\
   & =4\frac{\partial^2}{\partial z\partial\bar{z}}|f(z)\overline{f(z)}|^{p/2}                           \\
   & =4\frac{\partial^2}{\partial z\partial\bar{z}}|f(z)\overline{f(z)}|^{p/2}                           \\
   & = 4\frac{\partial}{\partial\bar{z}}\frac{p}{2}|f(z)|^{p/2-1}|f'(z)||\overline{f(z)}|^{p/2}          \\
   & = 4\cdot\frac{p}{2}|f(z)|^{p/2-1}|f'(z)|\cdot\frac{p}{2}|\overline{f(z)}|^{p/2-1}|\overline{f'(z)}| \\
   & =p^2|f(z)\overline{f(z)}|^{p/2-1}|f'(x)\overline{f'(x)}|                                            \\
   & = p^2|f(z)|^{p-2}|f'(x)|^2.
\end{align*}



\end{document}
