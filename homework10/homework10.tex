\documentclass[11pt,a4paper]{article}
\usepackage{../ma362}
\semester{Fall}
\year{2019}
\subtitlenumber{10}
\author{刘逸灏 (515370910207)}
\newcommand{\res}[1]{\mathop{\Res}\limits_{#1}}

\begin{document}
\maketitle

\section{六(一)/4}
\begin{problem}
求下列各积分之值:
\begin{enumerate}
  \item $\displaystyle\int_0^{2\pi}\frac{d\theta}{a+\cos\theta}\quad(a>1)$;
  \item $\displaystyle\int_0^{2\pi}\frac{dx}{(2+\sqrt{3}\cos x)^2}$.
\end{enumerate}
\end{problem}
\subsection*{(1)}
令$z=e^{i\theta}$, 则$d\theta=\dfrac{dz}{iz}$
$$\int_0^{2\pi}\frac{d\theta}{2a+\cos\theta}=\int_{|z|=1}\frac{2dz}{iz(a+z+z^{-1})}=\frac{2}{i}\int_{|z|=1}\frac{1}{z^2+2az+1}dz,$$
$$f(z)=\frac{1}{z^2+2az+1}=\frac{1}{(z+a-\sqrt{a^2-1})(z+a+\sqrt{a^2-1})}.$$
$-a+\sqrt{a^2-1}$为一阶极点, $-a-\sqrt{a^2-1}$为一阶极点, 只有$-a+\sqrt{a^2-1}$在圆$|z|<1$内.
$$\res{z=-a+\sqrt{a^2-1}}f(z)=\left.\frac{1}{z+a+\sqrt{a^2-1}}\right|_{z=-a+\sqrt{a^2-1}}=\frac{1}{2\sqrt{a^2-1}}.$$

由留数定理得
$$\int_0^{2\pi}\frac{d\theta}{2a+\cos\theta}=\frac{2}{i}\cdot 2\pi i\cdot \frac{1}{2\sqrt{a^2-1}}=\frac{2\pi}{\sqrt{a^2-1}}.$$

\subsection*{(2)}
令$z=e^{ix}$, 则$dx=\dfrac{dz}{iz}$
$$\int_0^{2\pi}\frac{dx}{(2+\sqrt{3}\cos x)^2}=\int_{|z|=1}\frac{4dz}{iz(4+\sqrt{3}z+\sqrt{3}z^{-1})^2}=\frac{4}{i}\int_{|z|=1}\frac{z}{(\sqrt{3}z^2+4z+\sqrt{3})^2}dz,$$
$$f(z)=\frac{z}{(\sqrt{3}z^2+4z+\sqrt{3})^2}=\frac{z}{(z+\sqrt{3})^2(\sqrt{3}z+1)^2}.$$
$-\sqrt{3}$为二阶极点, $-\dfrac{1}{\sqrt{3}}$为二阶极点, 只有$-\dfrac{1}{\sqrt{3}}$在圆$|z|<1$内.
$$\res{z=-\frac{1}{\sqrt{3}}}f(z)=\left.\left[\frac{z}{3(z+\sqrt{3})^2}\right]'\right|_{z=-\frac{1}{\sqrt{3}}}=\left.\frac{\sqrt{3}-z}{3(\sqrt{3}+z)^3}\right|_{z=-\frac{1}{\sqrt{3}}}=\frac{1}{2}.$$

由留数定理得
$$\int_0^{2\pi}\frac{dx}{(2+\sqrt{3}\cos x)^2}=\frac{4}{i}\cdot 2\pi i\cdot \frac{1}{2}=4\pi.$$

\section{六(一)/5}
\begin{problem}
求下列各积分:
\begin{enumerate}
  \addtocounter{enumi}{1}
  \item $\displaystyle\int_{-\infty}^{+\infty}\frac{x^2}{(x^2+a^2)^2}dx\quad(a>0)$;
        \addtocounter{enumi}{1}
  \item $\displaystyle\int_0^{+\infty}\frac{x\sin mx}{x^4+a^4}dx\quad(m>0,a>0)$.
\end{enumerate}
\end{problem}
\subsection*{(2)}
$$f(z)=\frac{z^2}{(z^2+a^2)^2}=\frac{z^2}{(z-ia)^2(z+ia)^2}.$$
$ia$为二阶极点, $-ia$为二阶极点, 只有$ia$在上半平面内.
$$\res{z=ia}=\left.\left[\frac{z^2}{(z+ia)^2}\right]'\right|_{z=ia}=\left.-\frac{2az}{(a-iz)^3}\right|_{z=ia}=-\frac{i}{4a}.$$
由定理6.7得
$$\frac{x^2}{(x^2+a^2)^2}dx=2\pi i\cdot-\frac{i}{4a}=\frac{\pi}{2a}.$$

\subsection*{(4)}
被积函数是偶函数, 故
$$\int_0^{+\infty}\frac{x\sin mx}{x^4+a^4}dx=\frac{1}{2}\int_{-\infty}^{+\infty}\frac{x\sin mx}{x^4+a^4}dx,$$
$$f(z)=\frac{ze^{imz}}{z^4+a^4}.$$
有四个一阶极点
$$a_k=ae^{\frac{\pi+2k\pi}{4}i},\quad (k=0,1,2,3),$$
$$\res{z=a_k}f(z)=\left.\frac{ze^{imz}}{(z^4+a^4)'}\right|_{z=a_k}=\left.\frac{e^{imz}}{4z^2}\right|_{z=a_k}=\frac{e^{ima_k}}{4a_k^2}.$$
$f(z)$在上半平面内只有两个极点$a_0$和$a_1$
$$\res{z=a_0}f(z)=\frac{e^{-\frac{\sqrt{2}ma}{2}+i\frac{\sqrt{2}ma}{2}}}{4a^2i}=\frac{e^{-\frac{\sqrt{2}ma}{2}}}{4a^2i}\left(\cos\frac{\sqrt{2}ma}{2}+i\sin\frac{\sqrt{2}ma}{2}\right),$$
$$\res{z=a_1}f(z)=\frac{e^{-\frac{\sqrt{2}ma}{2}-i\frac{\sqrt{2}ma}{2}}}{-4a^2i}=-\frac{e^{-\frac{\sqrt{2}ma}{2}}}{4a^2i}\left(\cos\frac{\sqrt{2}ma}{2}-i\sin\frac{\sqrt{2}ma}{2}\right).$$
由定理6.8得
$$\int_{-\infty}^{+\infty}\frac{xe^{imx}}{x^4+a^4}dx=2\pi i\cdot \left[\res{z=a_0}f(z)+\res{z=a_1}f(z)\right]=2\pi i\cdot \frac{e^{-\frac{\sqrt{2}ma}{2}}}{4a^2i}\cdot 2i\sin\frac{\sqrt{2}ma}{2}=i\frac{\pi}{a^2} e^{-\frac{\sqrt{2}ma}{2}}\sin\frac{\sqrt{2}ma}{2}.$$
且
$$\int_{-\infty}^{+\infty}\frac{xe^{imx}}{x^4+a^4}dx=\int_{-\infty}^{+\infty}\frac{x\cos mx}{x^4+a^4}dx+i\int_{-\infty}^{+\infty}\frac{x\sin mx}{x^4+a^4}dx.$$
故
$$\int_0^{+\infty}\frac{x\sin mx}{x^4+a^4}dx=\frac{\pi}{2a^2} e^{-\frac{\sqrt{2}ma}{2}}\sin\frac{\sqrt{2}ma}{2}.$$


\end{document}
