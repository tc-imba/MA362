\documentclass[11pt,a4paper]{article}
\usepackage{../ma362}
\semester{Fall}
\year{2019}
\subtitlenumber{2}
\author{刘逸灏 (515370910207)}

\begin{document}
\maketitle

\section{2.4/1}

设$z=x+yi$
$$\overline{e^z}=\overline{e^{x+yi}}=\overline{e^x(\cos y+i\sin y)}=e^x\overline{\cos y+i\sin y}=e^x(\cos y-i\sin y)=e^{x-yi}=e^{\bar{z}}.$$

\section{2.4/3}

设$z=x+yi$
$$e^z=e^{x+yi}=e^x(\cos y+i\sin y)=1,$$
$$e^x=|e^z|=1 \Longrightarrow x=0,$$
$$\cos y+i\sin y=\frac{e^z}{e^x}=1 \Longrightarrow y=2k\pi, \quad k\in Z.$$

故$z=2k\pi i$, $k=0,\pm1,\cdots$.

\section{2.4/4}

\subsection*{(i)}

由$f'(z)=f(z)$可得
$$[f(z)e^{-z}]'=f'(z)e^{-z}-f(z)e^{-z}=0.$$

根据习题2.2/1, $f(z)e^{-z}$为常数, 且已知$f(0)=1$, 可得
$$f(z)e^{-z}=f(0)e^0=1,$$
$$f(z)\equiv e^z.$$

\subsection*{(ii)}

由$f(z+\omega)=f(z)f(\omega)$可得
$$[f(z+\omega)]'=f'(z)f(\omega)+f(z)\cdot 0=f'(z)f(\omega).$$

当$z=0$时
$$f'(\omega)=f'(0)f(\omega)=f(\omega).$$

根据(i)可得$f(z)\equiv e^z$.

\section{2.4/7}

\subsection*{(i)}

由$f'(z)=e^{-f(z)}$可得
$$[e^{f(z)}-z]'=f'(z)e^{f(z)}-1=0.$$

根据习题2.2/1, $e^{f(z)}-z$为常数, 且已知$f(1)=0$, 可得
$$e^{f(z)}-z=e^{f(1)}-1=0,$$
$$f(z)\equiv \ln z.$$

\subsection*{(ii)}
由$f(z\omega)=f(z)+f(\omega)$可得
$$[f(z\omega)]'=\omega f'(z\omega)=f'(z)$$

当$z=1$时
$$\omega f'(\omega)=1,$$
$$f'(\omega)=\frac{1}{\omega},$$
$$f(\omega)=\ln\omega + C.$$

代入$f(1)=0$可得$f(z)\equiv \ln z$.

\section{2.4/8}

$\forall z_1\neq z_2$, $z_1, z_2\in B(0,1)$
$$f(z)=z^2+2z+3=(z+1)^2+2,$$
$$f(z_1)-f(z_2)=(z_1+1)^2-(z_2+1)^2=(z_1-z_2)(z_1+z_2+2).$$

$$|z_1+z_2+2|\geqslant 2-|z_1+z_2|,$$
$$|z_1+z_2|\leqslant|z_1|+|z_2|<2,$$
$$|z_1+z_2+2|>0.$$

故$f(z_1)-f(z_2)\neq 0$, $f(z)$在$B(0,1)$单叶.

\section{2.4/12}

\subsection*{(i)}

由$f'(z)=\mu\dfrac{f(z)}{z}$可得
$$[f(z)z^{-\mu}]'=f'(z)z^{-\mu}-\mu f(z)z^{-\mu-1}=z^{-\mu}\left(f'(z)-\mu\frac{f(z)}{z}\right)=0.$$

根据习题2.2/1, $f(z)z^{-\mu}$为常数, 且已知$f(1)=1$, 可得
$$f(z)z^{-\mu}=f(1)\cdot 1^{-\mu}=1,$$
$$f(z)\equiv z^{\mu}=e^{\mu\ln z}=|z|^\mu e^{i\mu\arg z}e^{i2k\pi\mu}.$$

取$k=0$时的单叶性域
$$f(z)\equiv |z|^\mu e^{i\mu\arg z}.$$

\subsection*{(ii)}

由$f(z\omega)=f(z)f(\omega)$可得
$$[f(z\omega)]'=\omega f'(z\omega)=f'(z)f(\omega)+f(z)\cdot0=f'(z)f(\omega).$$

当$z=1$时
$$\omega f'(\omega)=\mu f(\omega),$$
$$f'(\omega)=\mu\frac{f(\omega)}{\omega}.$$

根据(i)可得$f(z)\equiv |z|^\mu e^{i\mu\arg z}$.

\section{2.4/17}

$\forall z_1\neq z_2$, $z_1, z_2\in D$, 即证
$$\cos z_1-\cos z_2=-2\sin\frac{z_1+z_2}{2}\sin\frac{z_1-z_2}{2}\neq 0,$$
$$\sin z_1-\sin z_2=2\cos\frac{z_1+z_2}{2}\sin\frac{z_1-z_2}{2}\neq 0.$$

等价于当$k\in Z$
\begin{align*}
  \cos z\text{ 在}D\text{中单叶}: & \quad z_1+z_2\neq 2k\pi,\quad z_1-z_2\neq 2k\pi,     \\
  \sin z\text{ 在}D\text{中单叶}: & \quad z_1+z_2\neq (2k+1)\pi,\quad z_1-z_2\neq 2k\pi.
\end{align*}


\subsection*{(i)}

$$D=\{x\in \mathbf{C}:\theta_0<\Re z<\theta_0+\pi\}.$$
$$\Re(z_1-z_2)\in(-\pi,0)\cup(0,\pi) \Longrightarrow z_1-z_2\neq 2k\pi.$$
$$\Re(z_1+z_2)\in(2\theta_0,2\theta_0+2\pi)\Longrightarrow \left\{\begin{aligned}
     & z_1+z_2\neq 2k\pi\ \text{当且仅当}\ \theta_0=k\pi                              \\
     & z_1+z_2\neq (2k+1)\pi\ \text{当且仅当}\ \theta_0=\left(k+\frac{1}{2}\right)\pi\end{aligned}\right..
$$

故$\theta_0=k\pi$时, $\cos z$在$D$中单叶; $\theta_0=\left(k+\dfrac{1}{2}\right)\pi$时, $\sin z$在$D$中单叶.

\subsection*{(ii)}

$$D=\{x\in \mathbf{C}:\theta_0<\Re z<\theta_0+2\pi,\Im z>0\}.$$
$$\Re(z_1-z_2)\in(-2\pi,0)\cup(0,2\pi) \Longrightarrow z_1-z_2\neq 2k\pi.$$
$$\Im(z_1+z_2)=\Im z_1+\Im z_2 > 0\Longrightarrow z_1+z_2\neq k\pi.$$

故$\cos z$和$\sin z$在$D$中单叶.

\subsection*{(iii)}

$$D=\{x\in \mathbf{C}:\theta_0<\Re z<\theta_0+2\pi,\Im z<0\}.$$
$$\Re(z_1-z_2)\in(-2\pi,0)\cup(0,2\pi) \Longrightarrow z_1-z_2\neq 2k\pi.$$
$$\Im(z_1+z_2)=\Im z_1+\Im z_2 < 0\Longrightarrow z_1+z_2\neq k\pi.$$

故$\cos z$和$\sin z$在$D$中单叶.

\end{document}
